\documentclass[11pt, a4paper, titlepage]{article}
\usepackage[left=2cm, text={17cm, 24cm}, top=3cm]{geometry}
\usepackage[utf8]{inputenc}
\usepackage[czech]{babel}
\usepackage[hidelinks]{hyperref}
\usepackage{pdfpages}

% \setlength\parindent{0pt}

%%%%%%%%%%%%%%%%%%%%%%%%%%%%%%%%%%%%%%%%%%%%%%%%%%%%%%%%%%%%%%%%%%%%%

\begin{document}

\begin{center}
    \Large VUT FIT FLP 2019/2020
    \bigskip

    \Large Zadání online půlsemestrálního testu
    \bigskip

    \large \textit{Jedná se o programy v Haskellu, příklady mohou sloužit k procvičení.}
\end{center}

\section*{a)}

V~jazyku Haskell nadefinujte datový typ pro reprezentaci aritmetických výrazů s~operacemi násobení a sčítání nad proměnnými -- typ proměnné je dopředu neznámý. Definujte funkci tr, která má jako parametr výraz definovaný dle vašeho typu. Na základě vlastnosti asociativity a komutativity operace sčítání pro reálná čísla upraví datovou reprezentaci tak, aby při vyhodnocení post-order průchodem bylo sčítání jednotlivých operandů prováděno v~maximální míře zleva doprava. Pokud pro pořadí použijeme v~linearizovaném zápisu závorky, tak by např. takovýto výraz $(a+b)+(c+d)$ byl transformován na $((a+b)+c)+d$, nebo odbobný až na pořadí proměnných, nicméně význam výrazu se změnit nesmí. Volně využijte obsah Prelude.

\section*{b)}

V~jazyku Haskell nadefinujte datový typ pro reprezentaci aritmetických výrazů s~operacemi násobení a sčítání nad celými čísly a proměnnými -- typ proměnné je dopředu neznámý. Definujte funkci eo, která využije vlastnosti čísla jedna vzhledem k~operaci násobení a co nejvíce zjednoduší výraz, který je dán jako parametr a ten zjednodušený vrátí jako výsledek. Dbejte na to "co nejvíce". Volně využijte obsah Prelude.

\section*{c)}

V~jazyku Haskell nadefinujte datový typ pro reprezentaci booleovských výrazů s~operacemi logického součtu, součinu i negace nad booleovskými hodnotami a proměnnými -- typ proměnné je dopředu neznámý. Definujte funkci dm, která má dva parametry, tabulku symbolů a booleovský výraz dle vašeho typu. Tabulka symbolů je seznam dvojic proměnná, hodnota. Funkce dm odstraní dvojí negaci, aplikuje de Morganova pravidla a pro proměnné obsažené v~tabulce dosadí jejich hodnoty. Vše s~maximální účinností. Volně využijte obsah Prelude.

\section*{d)}

V~jazyku Haskell nadefinujte datový typ pro reprezentaci aritmetických výrazů s~operacemi násobení a sčítání nad celými čísly a proměnnými -- typ proměnné je dopředu neznámý. Definujte funkci ez, která využije vlastnosti čísla nula vzhledem k~operaci násobení a co nejvíce zjednoduší výraz, který je dán jako parametr a ten zjednodušený vrátí jako výsledek. Dbejte na to "co nejvíce". Volně využijte obsah Prelude.

\section*{e)}

V~jazyku Haskell definujte typ pro reprezentaci vyhledávacího binárního stromu nad klíčem a hodnotou dopředu neznámého typu. Definujte funkci put, která pro zadaný klíč a hodnotu provede vložení hodnoty a úpravu daného stromu dle vašeho typu. Pokud je více hodnot spojených se stejným klíčem, tak je ukládá do seznamu k~danému klíči, ale neduplikuje je, tedy každá hodnota je v~seznamu jen 1x!

\section*{f)}

V~jazyku Haskell definujte typ pro reprezentaci vyhledávacího binárního stromu nad klíčem a hodnotou dopředu neznámého typu. Definujte funkci ins, která pro zadaný klíč a hodnotu provede vložení hodnoty a úpravu daného stromu dle vašeho typu. Situaci, že pro daný klíč již hodnota existuje, vhodně ošetřete, úprava stromu se provádět nebude, vrátí se původní strom a informace o~tom, že vložení selhalo. Volně využijte obsah Prelude.

%%%%%%%%%%%%%%%%%%%%%%%%%%%%%%%%%%%%%%%%%%%%%%%%%%%%%%%%%%%%%%%%%%%%%

\end{document}
