\documentclass[11pt, a4paper, titlepage]{article}
\usepackage[left=2cm, text={17cm, 24cm}, top=3cm]{geometry}
\usepackage[utf8]{inputenc}
\usepackage[czech]{babel}
\usepackage{pdfpages}
\usepackage[obeyspaces]{url}
\usepackage{framed}
\usepackage[T1]{fontenc}
\usepackage{lmodern}
\usepackage{enumitem}

\setlength\parindent{0pt}

%%%%%%%%%%%%%%%%%%%%%%%%%%%%%%%%%%%%%%%%%%%%%%%%%%%%%%%%%%%%%%%%%%%%%

\begin{document}

\section*{IPK 2017/2018 - Projekt č.1}

\section*{Varianta termínu - Varianta 1: Klient-server pro získání informace o~uživatelích (Ryšavý)}
\bigskip


\subsection*{Společná část popisu}

Vypracujte jednoduchou klient-server aplikace dle uvedeného zadání.

\begin{itemize}
	\item Odevzdejte Váš projekt jako archív s~názvem odpovídající Vašemu loginu. Projekty odevzdané po termínu nebuou hodnoceny.
	\item Odevzdáváte zdrojové soubory projektu
	\item Projekt musí obsahovat soubor Makefile pro přeložení pomocí make (gmake) (\url{http://www.linuxsoft.cz/article.php?id_article=722})
	\item Projekt musí být přeložitelný pro uvedený systém (Linux, Unix)
	\item Součástí projektu je také dokumentace  ve formátu plain text, Markdown, nebo PDF.
\end{itemize}


\subsection*{Popis varianty}

\subsubsection*{\textbf{Zadání}}

Vašim úkolem je:

\begin{enumerate}
	\item Seznamte se s~kostrami kódů pro programování klientských a serverových síťových aplikací (přednáška třetí) za použití BSD socketů. Navrhněte vlastní aplikační protokol realizující přenos informací o~uživatelích na straně serveru a relevantní informace k~projektu uveďte v~dokumentaci. (8 bodů)
	\item Naprogramujte jak klientskou, tak serverovou aplikaci v~C/C++ realizující zprostředkování informace o~uživatelích na serveru. (12 bodů)
\end{enumerate}


\subsubsection*{\textbf{Konvence odevzdaného zip archivu xlogin00.zip}}

\begin{itemize}
	\item dokumentace.pdf - výstupy zadání [1]
	\item readme - základní informace, případná omezení projektu
	\item Makefile
	\item *.c, *.cpp, *.cc, *.h - zdrojové a hlavičkové soubory výstupů zadání [2]
\end{itemize}

\textbf{Ad [1]} \\
Navrhěte vlastní aplikační protokol, kterými poté spolu budou komunikovat klient a server z~bodu [2] tohoto zadání. Tento protokol bude sloužit pro přenos informací o~uživatelích na serveru. Informace o~uživatelích bude server získávat ze souboru /etc/passwd. \\

V~dobré dokumentaci se očekává následující: titulní strana, obsah, logické strukturování textu, výcuc relevantních informací z~nastudované literatury, popis aplikačního protokolu vhodnou formou, popis zajímavějších pasáží implementace, demonstrace činnosti implementovaných aplikací, normovaná bibliografie.

\textbf{Ad [2]} \\
Spuštění klienta předpokládá provedení pouze jedné operace. Komunikace mezi serverem a klientem bude spolehlivá. Server by měl umět obsloužit více požadavků. \\


\subsubsection*{\textbf{Konvence jména klientské aplikace a jejích povinných vstupních parametrů}}

\path{./ipk-client -h host -p port [-n|-f|-l] login}

\begin{itemize}
	\item host (IP adresa nebo fully-qualified DNS name) identifikace serveru jakožto koncového bodu komunikace klienta;
	\item port (číslo) cílové číslo portu;
	\item -n značí, že bude vráceno plné jméno uživatele včetně případných dalších informací pro uvedený login (User ID Info);
	\item -f značí, že bude vrácena informace o~domácím adresáři uživatele pro uvedený login (Home directory);
	\item -l značí, že bude vrácen seznam všech uživatelů, tento bude vypsán tak, že každé uživatelské jméno bude na zvláštním řádku; v~tomto případě je login nepovinný. Je-li však uveden bude použit jako prefix pro výběr uživatelů.
	\item login určuje přihlašovací jméno uživatele pro výše uvedené operace.
\end{itemize}

\subsubsection*{např}

\begin{framed}
	\path{./ipk-client -h eva.fit.vutbr.cz -p 55555 -n rysavy} \\
	Rysavy Ondrej,UIFS,541141118
\end{framed}

\begin{framed}
	\path{./ipk-client -h eva.fit.vutbr.cz -p 55555 -f rysavy} \\
	/homes/kazi/rysavy
\end{framed}

\begin{framed}
	\path{./ipk-client -h host -p port -l} \\
	avahi \\
	bacova \\
	barabas \\
	... \\
	zezula \\
	zizkaj
\end{framed}

\begin{framed}
	\path{./ipk-client -h host -p port -l xa} \\
	xabduk00 \\
	xabdul03 \\
	... \\
	xaygun00
\end{framed}


\subsubsection*{\textbf{Konvence jména serverové aplikace a jejích povinných vstupních parametrů}}

\path{./ipk-server -p port}

\begin{itemize}
	\item port (číslo) číslo portu, na kterém server naslouchá na připojení od klientů.
\end{itemize}

\subsubsection*{např}

\begin{framed}
	./ipk-server -p 55555
\end{framed}


\subsubsection*{\textbf{Doporučení}}

\begin{itemize}
	\item V~případě výskytu chyby (neexistující login, problém v~komunikaci, chybné paramtery), bude ta vypsána na STDERR a na STDOUT nebude tisknuto nic.
	\item Výsledky vaší implementace by měly být co možná nejvíce multiplatformní mezi OS založenými na unixu, ovšem samotné přeložení projektu a funkčnost vaší aplikace budou testovány na serverech eva.fit.vutbr.cz a merlin.fit.vutbr.cz.
	\item Všechny implementované programy by měly být použitelné a řádně komentované. Pokud už přejímáte zdrojové kódy z~různých tutoriálů či příkladů z~Internetu (ne mezi sebou pod hrozbou ortelu disciplinární komise), tak je POVINNÉ správně vyznačit tyto sekce a jejich autory dle licenčních podmínek, kterými se distribuce daných zdrojových kódů řídí. V~případě nedodržení bude na projekt nahlíženo jako na plagiát!
	\item Pro snadné parsování vstupních argumentů se doporučuje použít funkci getopt().
	\item Ukončení serverové aplikace z~bodu [2] bude vázáno na SIGINT signál (tedy Ctrl+C).
	\item Projekt bude opravován ručně. Počítejte tedy s~nejzazším možným termínem oprav a reklamací určených garantem předmětu.
	\item Výsledky vaší implementace by měly být co možná nejvíce multiplatformní mezi OS založenými na unixu (dbejte na to zejména při vytváření Makefile).
	\item Do souboru Readme uveďte případná omezení funkcionality vašeho projektu - na dokumentovanou chybu se nahlíží v~lepším světle než na nedokumentovanou!
	\item Snažte se o~jednoduchý přehledný kód. Tento projekt nevyžaduje žádné složitosti.
\end{itemize}


\subsubsection*{\textbf{Literatura}}

\begin{itemize}
	\item O. Ryšavý, J. Ráb, IPK - BSD schránky - 3. přednáška
	\item K. Sollins, The TFTP Protocol (revision 2), https://tools.ietf.org/html/rfc1350
	\item J. Postel, J. Reynolds, FILE TRANSFER PROTOCOL (FTP),https://tools.ietf.org/html/rfc959
	\item P. Grygárek, Softwarová rozhraní systémů UNIX pro přístup k~síťovým službám, http://www.cs.vsb.cz/grygarek/LAN/sockets.html
	\item Understanding /etc/passwd, Online: https://www.cyberciti.biz/faq/understanding-etcpasswd-file-format/
\end{itemize}

\end{document}
