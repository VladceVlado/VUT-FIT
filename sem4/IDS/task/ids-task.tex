% VUT FIT 2BIT
% IDS 2017/2018
% Semester project
% Author: Vladimir Dusek, xdusek27
% Date: 2/5/2018
% File: ids-task.tex

%%%%%%%%%%%%%%%%%%%%%%%%%%%%%%%%%%%%%%%%%%%%%%%%%%%%%%%%%%%%%%%%%%%%%

\documentclass[11pt, a4paper, titlepage]{article}
\usepackage[left=2cm, text={17cm, 24cm}, top=3cm]{geometry}
\usepackage[utf8]{inputenc}
\usepackage[czech]{babel}
\usepackage{pdfpages}
\usepackage[obeyspaces]{url}
\usepackage{framed}
\usepackage[T1]{fontenc}
\usepackage{lmodern}
\usepackage{enumitem}

\setlength\parindent{0pt}

%%%%%%%%%%%%%%%%%%%%%%%%%%%%%%%%%%%%%%%%%%%%%%%%%%%%%%%%%%%%%%%%%%%%%

\begin{document}

\section*{IDS 2017/2018 - Semestrální projekt}
\bigskip

%%%%%%%%%%%%%%%%%%%%%%%%%%%%%%%%%%%%%%%%%%%%%%%%%%%%%%%%%%%%%%%%%%%%%

\subsection*{Popis projektu}

Cílem řešeného projektu je návrh a implementace relační databáze na zvolené téma.

Projekt sestává ze pěti částí, které se odevzdávají ve stanovených termínech do WISu:

\begin{enumerate}
	\item Datový model (ERD) a model případů užití – Datový model (ER diagram) zachycující strukturu dat, resp. požadavky na data v databázi, vyjádřený jako diagram tříd v notaci UML nebo jako ER diagram v tzv. Crow's Foot notaci a model případů užití vyjádřený jako diagram případů užití v notaci UML reprezentující požadavky na poskytovanou funkcionalitu aplikace používající databázi navrženého datového modelu. Datový model musí obsahovat alespoň jeden vztah generalizace/specializace (tedy nějakou entitu/třídu a nějakou její specializovanou entitu/podtřídu spojené vztahem generalizace/specializace; vč. použití správné notace vztahu generalizace/specializace v diagramu).

	\item SQL skript pro vytvoření základních objektů schématu databáze – SQL skript vytvářející základní objekty schéma databáze, jako jsou tabulky vč. definice integritních omezení (zejména primárních a cizích klíčů), a naplňující vytvořené tabulky ukázkovými daty. Vytvořené schéma databáze musí odpovídat datovému modelu z předchozí části projektu a musí splňovat upřesňující požadavky zadání.

	\item SQL skript s několika dotazy SELECT – SQL skript, který nejprve vytvoří základní objekty schéma databáze a naplní tabulky ukázkovými daty (stejně jako skript v bodě 2) a poté provede několik dotazů SELECT dle upřesňujících požadavků zadání.

	\item SQL skript pro vytvoření pokročilých objektů schématu databáze – SQL skript, který nejprve vytvoří základní objekty schéma databáze a naplní tabulky ukázkovými daty (stejně jako skript v bodě 2), a poté zadefinuje či vytvoří pokročilá omezení či objekty databáze dle upřesňujících požadavků zadání. Dále skript bude obsahovat ukázkové příkazy manipulace dat a dotazy demonstrující použití výše zmiňovaných omezení a objektů tohoto skriptu (např. pro demonstraci použití indexů zavolá nejprve skript EXPLAIN PLAN na dotaz bez indexu, poté vytvoří index, a nakonec zavolá EXPLAIN PLAN na dotaz s indexem; pro demostranci databázového triggeru se provede manipulace s daty, která vyvolá daný trigger; atp.).

	\item Dokumentace popisující finální schéma databáze – Dokumentace popisující řešení ze skriptu v bodě 4 vč. jejich zdůvodnění (např. popisuje výstup příkazu EXPLAIN PLAN bez indexu, důvod vytvoření zvoleného indexu, a výstup EXPLAIN PLAN s indexem, atd.).
\end{enumerate}
\medskip

%%%%%%%%%%%%%%%%%%%%%%%%%%%%%%%%%%%%%%%%%%%%%%%%%%%%%%%%%%%%%%%%%%%%%

\subsection*{Organizace projektu, řešení a obhajoby}

Studenti řeší projekt ve dvojici (v týmu). Každý z výsledků projektu musí být vypracován v souladu se studijními předpisy VUT a FIT a autorským zákonem, tj. zejména samostatně dvojicí studentů (týmem), která jej předkládá, jako svůj výsledek (viz čl. 11 Směrnice děkana FIT doplňující Studijní a zkušební řád VUT).
\medskip

Pro řešení studenti využívají čas volného využití v počítačových učebnách CVT nebo řeší na svých počítačích. Cvičící poskytují zájemcům konzultace. Kromě toho jsou zařazena do programu přednášek témata na podporu řešení projektů zaměřená na seznámení s prostředím, které budou studenti využívat k řešení projektů, např. databázový server Oracle 12c, vývojové prostředí Oracle SQL Developer a jazyk PL/SQL – vizte přednášky.
\medskip

Po první části projektu následuje obhajoba vytvořených modelů a po poslední části závěrečná obhajoba projektu. Cílem obhajob je zdůvodnit a diskutovat prezentované řešení a prokázat samostatnou práci (na obhajobě můžete být požádáni o vysvětlení či upřesnění kterékoliv části projektu).
\medskip

%%%%%%%%%%%%%%%%%%%%%%%%%%%%%%%%%%%%%%%%%%%%%%%%%%%%%%%%%%%%%%%%%%%%%

\subsection*{Témata projektu}

Je možné pokračovat na projektu z předmětu IUS. Pokud se studenti rozhodnou v projektu z IUS nepokračovat, pak si příslušné téma dvojice vybere ze seznamu témat.
\medskip

Zvolené téma není potřeba předem nikde hlásit, avšak všechny výsledky musí obsahovat název zvoleného téma (např. jako komentář v SQL skriptech, či vyznačený v diagramech nebo dokumentaci).
\medskip

%%%%%%%%%%%%%%%%%%%%%%%%%%%%%%%%%%%%%%%%%%%%%%%%%%%%%%%%%%%%%%%%%%%%%

\subsection*{Upřesňující požadavky zadání projektu}

\begin{itemize}
	\item V tabulkách databázového schématu musí být alespoň jeden sloupec se speciálním omezením hodnot, např. rodné číslo či evidenční číslo pojištěnce (RČ), identifikačí číslo osoby/podnikatelského subjektu (IČ), identifikační číslo lékařského pracoviště (IČPE), ISBN či ISSN, číslo bankovního účtu (vizte také tajemství čísla účtu), atp. Databáze musí v tomto sloupci povolit pouze platné hodnoty (implementujte pomocí CHECK integritního omezení nebo TRIGGER).

	\item V tabulkách databázového schématu musí být vhodná realizace vztahu generalizace/specializace určená pro čistě relační databázi, tedy musí být vhodně převeden uvedený vztah a související entity datového modelu do schéma relační databáze. Zvolený způsob převodu generalizace/specializace do schéma relační databáze musí být popsán a zdůvodněn v dokumentaci.

	\item SQL skript obsahující dotazy SELECT musí obsahovat konkrétně alespoň dva dotazy využívající spojení dvou tabulek, jeden využívající spojení tří tabulek, dva dotazy s klauzulí GROUP BY a agregační funkcí, jeden dotaz obsahující predikát EXISTS a jeden dotaz s predikátem IN s vnořeným selectem (nikoliv IN s množinou konstatních dat). U každého z dotazů musí být (v komentáři SQL kódu) popsáno srozumitelně, jaká data hledá daný dotaz (jaká je jeho funkce v aplikaci).

	\item SQL skript v poslední části projektu musí obsahovat vše z následujících
	\begin{itemize}
		\item vytvoření alespoň dvou netriviálních databázových triggerů vč. jejich předvedení, z toho právě jeden trigger pro automatické generování hotnot primárního klíče nějaké tabulky ze sekvence (např. pokud bude při vkládání záznamů do dané tabulky hodnota primárního klíče nedefinována, tj. NULL),

		\item vytvoření alespoň dvou netriviálních uložených procedur vč. jejich předvedení, ve kterých se musí (dohromady) vyskytovat alespoň jednou kurzor, ošetření výjimek a použití proměnné s datovým typem odkazujícím se na řádek či typ sloupce tabulky (\path{table_name.column_name%TYPE} nebo \path{table_name%ROWTYPE}),

		\item explicitní vytvoření alespoň jednoho indexu tak, aby pomohl optimalizovat zpracování dotazů, přičemž musí být uveden také příslušný dotaz, na který má index vliv, a v dokumentaci popsán způsob využití indexu v tomto dotazy (toto lze zkombinovat s EXPLAIN PLAN, vizte dále),

		\item alespoň jedno použití EXPLAIN PLAN pro výpis plánu provedení databazového dotazu se spojením alespoň dvou tabulek, agregační funkcí a klauzulí GROUP BY, přičemž v dokumentaci musí být srozumitelně popsáno, jak proběhne dle toho výpisu plánu provedení dotazu, vč. objasnění použitých prostředků pro jeho urychlení (např. použití indexu, druhu spojení, atp.), a dále musí být navrnut způsob, jak konkrétně by bylo možné dotaz dále urychlit (např. zavedením nového indexu), navržený způsob proveden (např. vytvořen index), zopakován EXPLAIN PLAN a jeho výsledek porovnán s výsledkem před provedením navrženého způsobu urychlení,

		\item definici přístupových práv k databázovým objektům pro druhého člena týmu,

		\item vytvořen alespoň jeden materializovaný pohled patřící druhému členu týmu a používající tabulky definované prvním členem týmu (nutno mít již definována přístupová práva), vč. SQL příkazů/dotazů ukazujících, jak materializovaný pohled funguje,
	\end{itemize}

	\item Řešení projektu může volitelně obsahovat také další prvky neuvedené explicitně v předchozích bodech či větší počet nebo složitost prvků uvedených. Takové řešení pak může být považováno za nadstandardní řešení a oceněno prémiovými body. Příkladem nadstandardního řešení může být řešení obsahující
	\begin{itemize}
		\item klientskou aplikaci realizovánou v libovolném programovacím jazyce, přičemž práce s aplikací odpovídá případům užití uvedených v řešení 1. části projektu – tedy aplikace by neměla pouze zobrazovat obecným způsobem tabulky s daty a nabízet možnost vkládání nových či úpravy a mazání původních dát, ale měla by odpovídat pracovním postupům uživatelů (např. knihovník po příchodu čtenáře žádá ID průkazky čtenáře, systém vypíše existující výpůjčky čtenáře s vyznačením případných pokut, knihovník má možnost označit jednolivé výpůjčky jako právě vrácené, případně inkasovat pokuty spojené s výpůjčkami, či přidat nové výpůjčky daného čtenáře),

		\item SQL dotazy a příkazy ukazující transakční zpracování, vč. jejich popisu a vysvětlení v dokumentaci – např. ukázka atomicity transakcí při souběžném přístupu více uživatelů/spojení k jedněm datům, ukázka zamykání, atp.
	\end{itemize}

\end{itemize}

Tip: pro ladění PL/SQL kódu v uložených procedurách či databázových triggerech můžete použít proceduru \path{DBMS_OUTPUT.put_line(...)} pro výstup na terminál klienta.
\medskip

%%%%%%%%%%%%%%%%%%%%%%%%%%%%%%%%%%%%%%%%%%%%%%%%%%%%%%%%%%%%%%%%%%%%%

\subsection*{Hodocení řešení projektu}

Za řešení splňující všechny požadavky definované v popisu částí projektu a upřesňujících požadavcích zadání projektu lze získat celkem 29 bodů. Za nadprůměrný výsledek obsahující další funkcionalitu či prvky nepožadované explicitně v zadání projektu lze získat dalších 5 bodů, zde je ponechán prostor pro iniciativu dvojice.

Celkově lze dosáhnout nejvýše 29 až 34 bodů. Za jednotlivé části řešení je následující počet bodů:

\begin{enumerate}
	\item Datový model (ERD) a model případů užití (s obhajobou) – max. 5 bodů

	\item SQL skript pro vytvoření základních objektů schématu databáze – max. 5 bodů

	\item SQL skript s několika dotazy SELECT – max. 5 bodů

	\item SQL skript pro vytvoření pokročilých objektů schématu databáze a Dokumentace popisující finální schéma databáze (s obhajobou) – max. 14 bodů, či až 19 bodů v případě naprůměrného výsledku.
\end{enumerate}

%%%%%%%%%%%%%%%%%%%%%%%%%%%%%%%%%%%%%%%%%%%%%%%%%%%%%%%%%%%%%%%%%%%%%

\end{document}
