\section{Úvod}

V projektu jsme modelovali a simulovali\footnote{\url{https://www.fit.vutbr.cz/study/courses/IMS/public/prednasky/IMS.pdf}, slide 8} provoz horské chaty Triglavski Dom na Kredarici\footnote{\url{https://en.wikipedia.org/wiki/Triglav_Lodge_at_Kredarica}}. Jedná se o horskou chatu v Julských Alpách ve Slovinsku. Chata se nachází pod vrcholem hory Triglav ve výšce 2515\,m\,n.\,m. a tvoří záchytný bod pro většinu turistů, kteří se vydají horu zdolat.

Chata je kompletně odříznutá od civilizace, nedá se k ní dostat jinak než pěšky nebo vrtulníkem. Pochopitelně nevede k ní ani vedení zvlášť vysokého napětí. U chaty se nachází dvě větrné turbíny a společně se solárními panely na střeše jsou jediným zdrojem elektrické energie pro chatu a její návštěvníky. Ta se využívá na svícení, topení, provoz restaurace a podobně.\footnote{\url{https://www.summitpost.org/kredarica-hut-triglavski-dom-na-kredarici/349588}}

V práci se zabýváme produkcí energie těchto dvou zdrojů v daných podmínkách. Zvažujeme možná rizika technické závady elektráren, vliv nepříznivého počasí, či životnost akumulátorů. Na základě modelu\footnote{\url{https://www.fit.vutbr.cz/study/courses/IMS/public/prednasky/IMS.pdf}, slide 7} a simulačních experimentů\footnote{\url{https://www.fit.vutbr.cz/study/courses/IMS/public/prednasky/IMS.pdf}, slide 9} bude ukázán poměr produkce a výdeje energie. Z toho vyvodíme jak často hrozí, že by chata byla bez elektřiny. Ať už vlivem špatného počasí, technických problémů, či nadměrné spotřebě energie.

Smyslem experimentů je demonstrovat, že pokud by se přistavěli další zdroje energie, zvýšila by se kapacita akumulátorů, či by probíhali častější technické kontroly, tak by se zvýšila stabilita elektřiny.



\subsection{Autoři a zdroje informací}

Simulační studii vypracovali Vladimír Dušek a Andrej Naňo. Při tvorbě projektu byly využity znalosti nabyté v předmětu IMS. Pro zpracování modelu bylo nutné nastudovat princip větrných a fotovoltaických elektráren, zejména vliv počásí na jejich chod a akumulátory vhodné pro skladování energie z těchto zdrojů. Dále zjistit kolik turistů chatu během roku navštěvuje a jak je její provoz energeticky náročný. Informace jsme hledali na internetu, v každém odstavci jsou pak přesné odkazy na články, na jejich základě jsme danou informaci získali. Také jsme využili osobních zkušeností jednoho z autorů, který chatu navštívil.



\subsection{Ověřování validity modelu}

Problém validity modelu\footnote{\url{https://www.fit.vutbr.cz/study/courses/IMS/public/prednasky/IMS.pdf}, slide 10} jsme řešili konfrontací předpokládaných hodnot s hodnotami vystupujících z našeho simulačního modelu.
Vstupní hodnoty jsme optimalizovali na základě simulačních experimentů s modelem. Po dosáhnutí relativně uspokojivě přesných výsledků jsme systém prohlásili za dostatečně validní.

Verifikace modelů\footnote{\url{https://www.fit.vutbr.cz/study/courses/IMS/public/prednasky/IMS.pdf}, slide 36} proběhla v kapitole ~\ref{chap:mapping}, kde jsme ukázali mapování struktur/entit abstraktního modelu na třídy simulačního modelu.
