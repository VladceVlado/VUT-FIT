% VUT FIT 3BIT
% ITU 2018/2019
% Project
% Authors:
%     Vladimir Dusek, xdusek27
%     Petr Knetl, xknetl00
%     Martin Hosala, xhosal00
% Date: 29/9/2018
% File: itu-assignment.tex

%%%%%%%%%%%%%%%%%%%%%%%%%%%%%%%%%%%%%%%%%%%%%%%%%%%%%%%%%%%%%%%%%%%%%

\documentclass[11pt, a4paper, titlepage]{article}
\usepackage[left=2cm, text={17cm, 24cm}, top=3cm]{geometry}
\usepackage[utf8]{inputenc}
\usepackage[czech]{babel}
\usepackage{pdfpages}
\usepackage[obeyspaces]{url}
\usepackage{framed}
\usepackage[T1]{fontenc}
\usepackage{lmodern}
\usepackage{enumitem}

\setlength\parindent{0pt}

%%%%%%%%%%%%%%%%%%%%%%%%%%%%%%%%%%%%%%%%%%%%%%%%%%%%%%%%%%%%%%%%%%%%%

\begin{document}

\section*{ITU 2018/2019 - Projekt}
\bigskip

%%%%%%%%%%%%%%%%%%%%%%%%%%%%%%%%%%%%%%%%%%%%%%%%%%%%%%%%%%%%%%%%%%%%%

\subsection*{Individuální zadání projektu}

Jeden projekt v délce trvání po celou dobu kursu.

Ačkoliv zadaná témata popisují většinou nějakou funkci programu, cílem projektu je navrhnout, zrealizovat, otestovat a vyhodnotit uživatelské rozhraní. Hodnocení projektu bude zaměřeno na tyto atributy. U projektu do ITU nebude hodnoceno, co všechno a jak dobře program umí, ale jak je rozhraní promyšlené, jak jsou schopnosti programu zpřístupněny uživateli, postup a výsledky testování a vyhodnocení.

Projekty mohou řešit týmy o 1-3 členech. Při odevzdání projektu je nutné přesně popsat přínos jednotlivých členů týmu na řešení projektu. Nastanou-li problémy v týmu během řešení projektu, včas konzultujte problém s učitelem. Kvalitu řešení projektu testujte pomocí dobrovolníků, jejichž zpětné vazby citujte v dokumentaci. Je třeba odevzdat zdrojové texty, spustitelný projekt (jsou-li předmětem řešení), zprávu v el. podobě a provést demonstraci. Programovací jazyk uvedený v zadání je doporučení, nikoliv podmínka.

%%%%%%%%%%%%%%%%%%%%%%%%%%%%%%%%%%%%%%%%%%%%%%%%%%%%%%%%%%%%%%%%%%%%%

\subsection*{Vlastní zadání}

Vlastní zadání jsou vítána, jedná-li se o nové a zajímavé téma. Máte-li nějaký nápad, nejdříve se podívejte, není-li podobné již vypsanému tématu a k tomu se případně přihlašte. V případě, že takové ani podobné zadání neexistuje, pošlete e-mail s návrhem zadání a seznam členů týmu pro individuální posouzení.

%%%%%%%%%%%%%%%%%%%%%%%%%%%%%%%%%%%%%%%%%%%%%%%%%%%%%%%%%%%%%%%%%%%%%

\subsection*{Odevzdání}

Odevzdání projektu bude probíhat elektronicky a bude doplněno povinnou demonstrací výsledků v termínech cvičení všemi členy týmu. Do IS se odevzdávají dva soubory, jeden *.zip s pouze zdrojovými soubory a jeden *.pdf soubor s dokumentací (každý max. velikost 2MB, pozor, limit nelze navýšit!). Odevzdávejte pouze Vámi vytvořené soubory, na převzaté knihovny se odkazujte v dokumentaci s příslušným popisem jejich začlenění do Vašeho projektu.

%%%%%%%%%%%%%%%%%%%%%%%%%%%%%%%%%%%%%%%%%%%%%%%%%%%%%%%%%%%%%%%%%%%%%

\subsection*{Prezentace a demonstrace}

Prezentace a demonstrace je povinná a je možná až po elektronickém odevzdání. Na termín prezentace se musí týmy registrovat. Při demonstraci se zaměřte především na to, čeho jste chtěli v projektu dosáhnout a pak na způsob a výsledky testování, na kolik se Vám to podařilo.

%%%%%%%%%%%%%%%%%%%%%%%%%%%%%%%%%%%%%%%%%%%%%%%%%%%%%%%%%%%%%%%%%%%%%

\subsection*{Hodnocení}

Orientační rozdělení hodnocení projektu:
\begin{itemize}
	\item implementace - 55\%
	\item prezentace - 5\%,
	\item technická zpráva (pouze v el. podobě) - 15\%,
	\item testování a vyhodnocení - 10\%,
	\item průběžná činnost na projektu - 15\%
	\item maximálně 55b.
\end{itemize}

%%%%%%%%%%%%%%%%%%%%%%%%%%%%%%%%%%%%%%%%%%%%%%%%%%%%%%%%%%%%%%%%%%%%%

\subsection*{Technická zpráva}

Do technické zprávy (na rozdíl od prezentace) uveďte vše, co se týká vypracování projektu, jeho implementace a testování. Dokumentujte informační zdroje, ze kterých bylo čerpáno při řešení, vlastní myšlenky a přínos. Nepopisujte všeobecně známé věci a triviality. Podrobně vyjmenujte použité knihovny ve Vašem řešení a postup při jejich kompilaci s Vaší implementací. Šablona k technické zprávy je v souborech k předmětu.

%%%%%%%%%%%%%%%%%%%%%%%%%%%%%%%%%%%%%%%%%%%%%%%%%%%%%%%%%%%%%%%%%%%%%

\subsection*{Pravidla vypracování projektů}

Studenti ve své práci musí pracovat samostatně a tvůrčím způsobem. Za porušení této zásady se považuje zejména reverzní inženýrství (disasebmling, dekompilace a podobné postupy), kopírování příkladů řešení, hotových řešení nebo obdobných podkladů, které jsou zveřejněny nebo jsou studentům jinak dostupné (jedná se o kopírování celých řešení nebo jejich tak velkých částí, že jejich okopírování vede k funkčně shodnému nebo velmi obdobnému řešení zadání), společná práce na zadání ve skupinách tak, že její výsledky jsou potom odevzdávány jako řešení jednotlivce (jednotlivců), pokud to není v zadání přímo požadováno nebo povoleno (diskuse ve skupině a/nebo společné řešení dílčích částí je povoleno). Studenti se musí zdržet jednání, které je v rozporu s dobrými mravy a které by mohlo vést k obcházení skutečného způsobu "řešení" zadání v duchu těchto zadání jimi samotnými nebo jinými studenty. Pokud student(i) poruší výše uvedená pravidla, může mu hodnocení projektu být sníženo až na 0 bodů.

%%%%%%%%%%%%%%%%%%%%%%%%%%%%%%%%%%%%%%%%%%%%%%%%%%%%%%%%%%%%%%%%%%%%%

\subsection*{Varianta termínu - Kapacita krátkodobé paměti}

Stanovte metodiku pro vyhodnocení kapacity krátkodobé paměti (i pro různé druhy informace). Metodika by měla zahrnovat zatížení testované osoby i jinou činností. Napište jednoduchý program pro vyhodnocování kapacity krátkodobé paměti. Zaměřte se na intuitivnost a komfort ovládání. Realizujte s využitím technologií HTML/CSS/MySQL/PHP/Javascript/AJAX/jQuery.

%%%%%%%%%%%%%%%%%%%%%%%%%%%%%%%%%%%%%%%%%%%%%%%%%%%%%%%%%%%%%%%%%%%%%

\end{document}
