% VUT FIT MITAI
% KRY 2020/2021
% Project 1: Vigenere cipher
% Author: Vladimir Dusek
% Login: xdusek27

%%%%%%%%%%%%%%%%%%%%%%%%%%%%%%%%%%%%%%%%%%%%%%%%%%%%%%%%%%%%%%%%%%%%%

\documentclass[11pt, a4paper, titlepage]{article}
\usepackage[left=2cm, text={17cm, 24cm}, top=3cm]{geometry}
\usepackage[utf8]{inputenc}
\usepackage[czech]{babel}
\usepackage{pdfpages}
\usepackage[T1]{fontenc}
\usepackage{amsmath}
\usepackage{graphicx}
\usepackage{float}
\usepackage[obeyspaces]{url}

\begin{document}

%%%%%%%%%%%%%%%%%%%%%%%%%%%%%%%%%%%%%%%%%%%%%%%%%%%%%%%%%%%%%%%%%%%%%

\begin{center}
    \Large Kryptografie (KRY)
    \bigskip

    \Large Dokumentace k~$1.$ projektu: \textit{Lámání Vigenerovy šifry}
    \bigskip

    \Large Vladimír Dušek, xdusek27
    \bigskip
\end{center}

%%%%%%%%%%%%%%%%%%%%%%%%%%%%%%%%%%%%%%%%%%%%%%%%%%%%%%%%%%%%%%%%%%%%%

\section{Úvod}\label{sec_intro}

Cílem projektu je prolomení Vigenerovy šifry.
Jedná se o~historickou symetrickou polyalfabetickou substituční šifru.
Vigenerova šifra šifruje text pomocí série různých Caesarových šifer v~závislosti na písmenech klíče.
Princip lámání spočívá v~odhadnutí délky hesla pomocí Kasiského testu
a jeho upřesnění pomocí Friedmanova testu.
Pokud je zjištěna délka hesla, lze problém rozdělit na $n$ textů zašifrovaných Caesarovou šifrou
(kde $n$ je délka klíče).
Caesarovu šifru lze pak prolomit hrubou silou.

%%%%%%%%%%%%%%%%%%%%%%%%%%%%%%%%%%%%%%%%%%%%%%%%%%%%%%%%%%%%%%%%%%%%%

\section{Kasiského test}\label{sec_kasiski}

Kasiského test spočívá v~hledání shodných posloupností písmen v~šifrovaném textu.
Využívá toho, že je velmi pravděpodobné, že tyto posloupnosti byly zašifrovány stejnou částí klíče
a tedy i původní význam je shodný.
Tyto posloupnosti mohou být různě dlouhé.
Čím delší vstupní text, tím delší posloupnosti je možné hledat.
U~delších posloupností je nižší pravděpodobnost, že se jedná pouze o~náhodu, a že je text skutečně
zašifrován shodnou částí klíče.\footnote{\path{https://en.wikipedia.org/wiki/Kasiski_examination}}

Po nalezení daného počtu takovýchto posloupností,
jsou spočítány jejich jednotlivé vzájemné vzdálenosti.
Ze vzdáleností je spočítán největší společný dělitel, který je pravděpodná délka klíče.
U~každé vzdálenosti jsou zaznamenány její četnosti.
Pokud je četnost pod určitou mezí, tak vzdálenost není zahrnuta
do počítání největšího společného dělitele.
Tím lze částečně eliminovat chybné vzdálenosti.

Vzhledem k~časovému omezení běhu programu je hledáno maximálně 2000 shodných posloupností.
Délka posloupnosti je určována dynamicky v~závislosti na délce zašifrovaného textu.
Do $5 000$ znaků jsou hledány trojice, do $25 000$ znaků čtveřice a cokoliv výše hledá pětice.
Vzdálenosti, které nemají aspoň $3$ výskyty jsou eliminovány.

%%%%%%%%%%%%%%%%%%%%%%%%%%%%%%%%%%%%%%%%%%%%%%%%%%%%%%%%%%%%%%%%%%%%%

\section{Friedmanův test}\label{sec_friedman}

Friedmanův test využívá tzv. index koincidence, který měří rovnoměrnost výskytů
jednotlivých písmen v~textu.
Využívá toho, že v~jednotlivých jazycích jsou frekvence různých písmen odlišné.
Pokud je známo, že zašifrovaný text je v~Angličtině, lze této skutečnosti
využít.\footnote{\path{https://en.wikipedia.org/wiki/Vigenere_cipher}}

Pokud známe pravděpodobnost, že dva náhodně vybrané znaky z~textu jsou shodné,
tak můžeme délku klíče odhadnout vzorcem~\ref{eq_1}.
Pro Angličtinu je to asi $\kappa_{p} = 0,067$ a pro čistě náhodný text $\kappa_{r} = 0,0385$.

\begin{equation}
    friedman\_test = \frac{\kappa_{p} - \kappa_{r}}{\kappa_{o} - \kappa_{r}}
    \label{eq_1}
\end{equation}

Parametr $\kappa_{o}$ je spočítán podle rovnice~\ref{eq_2}.
$N$ je délka textu, $c$ je počet znaků abecedy a $n_i$ je počet výskytů
daného znaku v~zašifrovaném textu.

\begin{equation}
    \kappa_{o} = \frac{ \sum_{i=1}^c n_i (n_i - 1) }{N (N - 1)}
    \label{eq_2}
\end{equation}

Pokud se výsledek Friedmanova testu výrazně liší od výsledku Kasiského testu, je proveden další výpočet.
Hranice pro výskyt vzdálenosti je inkrementována, vzdálenosti co novou hranici nesplňují jsou eliminovány
a je znovu spočítán největší společný dělitel (pokud ještě nějaké vzdálenosti zbyly).
Tímto lze odhadnutou délku hesla upřesnit, pokud Kasiského test vzal v~potaz chybnou vzdálenost.

%%%%%%%%%%%%%%%%%%%%%%%%%%%%%%%%%%%%%%%%%%%%%%%%%%%%%%%%%%%%%%%%%%%%%

\section{Prolomení klíče}\label{sec_key}

Pokud je známá délka klíče $n$, je možné text rozdělit na $n$ částí, kde každá část
je šifrovaná Caesarovou šifrou.
Vzniká $n$ podtextů, každý šifrovaný jedním znakem.
Tyto podtexty lze prolomit útokem hrubou silou.
Jsou vyzkoušeny každý ze 26 možných posuvů a jazykovou analýzou je zjištěn ten nejpravděpodobnější,
tedy ten co nejvíce odpovídá Angličtině.
Po aplikaci stejného principu na každý z~podtextů,
jsme získali pravděpodobný klíč k~původnímu textu zašifrovaným Vigenerovou šifrou.

Nejpravděpodobnější posuv je spočítán pomocí metody podobné indexu koincidence.
Tentokrát jako pravděpodobnost výskytu každého jednotlivého znaku v~daném jazyce.

\begin{equation}
    M_g = \sum_{i=0}^{25} \frac{ p_i f_{i + g} }{n}
    \label{eq_3}
\end{equation}

Přesný výpočet je zachycen na rovnici~\ref{eq_3}.
Kde $n$ je délka podřetězce, $g$ je jeden z~možných posuvů v~rámci Caesarovy šifry,
$f_i+g$ značí frekvence písmen v~podřetězci a $p_i$ frekvence písmen v~Angličtině.
Pokud $g$ je správny posuv, pak bude hodnota $M_g$ odpovídat zhruba hodnotě
$0,065$.\footnote{\path{https://wis.fit.vutbr.cz/FIT/st/cfs.php.cs?file=\%2Fcourse\%2FKRY-IT\%2Flectures\%2FKRY01_Klas_MNG.pdf}}

%%%%%%%%%%%%%%%%%%%%%%%%%%%%%%%%%%%%%%%%%%%%%%%%%%%%%%%%%%%%%%%%%%%%%

\section{Testování}\label{sec_test}

Pro účely testování byl napsán bashový skript \path{test.sh}, který program testuje na
náhodných anglických textech o~délce 3500 -- 50 000 znaků.
Texty jsou šifrovány náhodnými klíči v~délce 1 -- 200.

Skript kontroluje návratové kódy, výsledky testů, odhadnuté heslo a délku běhu programu,
která nesmí přesáhnout 30 sekund. Na konci je vypsán report.

\begin{center}
    \path{$ ./test.sh}
\end{center}

%%%%%%%%%%%%%%%%%%%%%%%%%%%%%%%%%%%%%%%%%%%%%%%%%%%%%%%%%%%%%%%%%%%%%

\end{document}
