%%%%%%%%%%%%%%%%%%%%%%%%%%%%%%%%%%%%%%%%%%%%%%%%%%%%%%%%%%%%%%%%%%%%%%%%%%%%%%%%

\section{Výsledky}
\label{sec_vysledky}

V~kapitole~\ref{sec_analyza} bylo ukázáno, že pokud výpočetní výkon poolu překročí určitou hranici, pool by mohl zvýšit svoji celkovou odměnu zvolením strategie sobeckého těžení (teorém~\ref{theorem_1}). V~takovém případě by racionální těžaři měli preferovat připojit se k~sobeckému poolu, kde mohou zvýšit svoji odměnu. Navíc v~zájmu členů sobeckého poolu je přijímat nové členy, jelikož i to zvyšuje jejich odměny (teorém~\ref{theorem_2}).

Sobecký pool tímto způsobem bude nabývat na velikosti, až se stane majoritní ($>0,5$). Jakmile se nějaký pool stane majoritním, získává absolutní kontrolu nad blockchainem. Strategie sobeckého těžení se stává nerelevantní, jelikož zbytek sítě již není rychlejší než pool. V~takovém případě majoritní pool může pohodlně ignorovat bloky zbytku sítě a vytěžit všechny bloky sám. Tím získá odměny ze všech bloků. Může také rozhodovat jaké transakce budou vytěženy a jaké ne -- cenzura transakcí. V~tento moment přestává bitcoinová síť být decentralizovaná.

V~článku~\cite{bib_paper} v~kapitole $6$ je navrhnuto, jak by bitcoinový protokol mohl být upraven a být více rezistentní proti strategie sobeckého těžení. Jedná se však pouze o~posunutí hranice, kdy se sobecké těžení vyplatí, ne kompletní vyřešení problému.

%%%%%%%%%%%%%%%%%%%%%%%%%%%%%%%%%%%%%%%%%%%%%%%%%%%%%%%%%%%%%%%%%%%%%%%%%%%%%%%%
