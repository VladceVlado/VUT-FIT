%%%%%%%%%%%%%%%%%%%%%%%%%%%%%%%%%%%%%%%%%%%%%%%%%%%%%%%%%%%%%%%%%%%%%%%%%%%%%%%%

\section{Úvod}
\label{sec_uvod}

Bitcoin je nestátní, decentralizovaný, digitální finanční systém založený na kryptografii. Ze své podstaty funguje bez jakékoliv centrální autority. Narozdíl od státních \textit{fiat} měn\footnote{Státní peníze vznikající z~příkazu. Z~lanského \textit{fiat} = budiž. Nejčastěji používán jako termín pro papírové peníze, nekryté drahými kovy~\cite{bib_odluka_penez}.}, které jsou řízeny centrálními bankami. Bitcoinová síť je čistě \textit{peer-to-peer}, kde si mezi sebou uživatelé vyměňují hodnotu bez potřeby jakéhokoliv prostředníka~\cite{bib_white_paper}. Je navržen tak, aby nikdo, ani autor, jiní jednotlivci, skupiny či státy, nemohl měnu ovlivňovat, padělat, konfiskovat, devalvovat, cenzurovat nebo ovlivňovat růst peněžní zásoby (ta je pevně definovaná a všem dopředu známá). V~protokolu neexistuje žádný centrální ani nijak privilegovaný bod, všichni uživatelé operují jako uzly na stejné úrovni.

Bitcoin byl představen 31. října 2008, kdy člověk nebo skupina lidí pod pseudonymem Satoshi Nakamoto zaslal e-mail kryptografickému mailing listu s~odkazem na \textit{white paper}~\cite{bib_mailing_list}. Bitcoinová síť pak byla spuštěna 9. ledna roku 2009, kdy byl vytěžen první blok~\cite{bib_first_block}. Ke konci roku 2020 se  tržní kapitalizace Bitcoinu pohybuje kolem $500$ miliard amerických dolarů~\cite{bib_market_cap}. Výpočetní výkon (\textit{hash rate}) celé sítě činí zhruba $130 \times 10^{18}$\,H/s (\textit{hashes per second})~\cite{bib_hash_rate}.

Článek \textit{Majority is not Enough: Bitcoin Mining is Vulnerable}~\cite{bib_paper} ukazuje, že v~bitcoinovém protokolu nejsou ekonomické incentivy těžařů nastaveny korektně. Představuje strategii tzv. sobeckého těžení (z~anglického \textit{Selfish Mining Strategy}), která vede k~většímu zisku, než by mělo těžaři náležet na základě jeho výpočetní síly pokud by se choval tak, jak protokol předpokládá.

V~kapitole~\ref{sec_bitcoin} jsou vysvětleny základy fungování Bitcoinu jakožto decentralizovaného finančního systému. Ty jsou nutné k~pochopení strategie sobeckého těžení a následnému sestavení modelu. Ten je formalizován v~kapitole~\ref{sec_model} a analyzován v~kapitole~\ref{sec_analyza}. Výsledky analýzy jsou vyhodnoceny v~kapitole~\ref{sec_vysledky} a na závěr jsou stručně okomentovány.

%%%%%%%%%%%%%%%%%%%%%%%%%%%%%%%%%%%%%%%%%%%%%%%%%%%%%%%%%%%%%%%%%%%%%%%%%%%%%%%%
